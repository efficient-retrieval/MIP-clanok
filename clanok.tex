% Metódy inžinierskej práce

% TODO twoside
\documentclass[10pt,a4paper]{article}

%\usepackage[slovak]{babel}
%\usepackage[T1]{fontenc}
%\usepackage[IL2]{fontenc} % lepšia sadzba písmena Ľ než v T1
\usepackage[utf8]{inputenc}
\usepackage{graphicx}
\usepackage{url} % príkaz \url na formátovanie URL
\usepackage{hyperref} % odkazy v texte budú aktívne (pri niektorých triedach dokumentov spôsobuje posun textu)
\usepackage{lipsum}

\usepackage{cite}
%\usepackage{times}

\pagestyle{headings}

\title{Getting information to the user optimally\thanks{Semestrálny projekt v predmete Metódy inžinierskej práce, ak. rok 2023/24, vedenie: Meno Priezvisko}} % TODO: meno a priezvisko vyučujúceho na cvičeniach

\author{Samuel Tvrdoň\\[2pt]
	{\small Slovenská technická univerzita v Bratislave}\\
	{\small Fakulta informatiky a informačných technológií}\\
	{\small \texttt{...@stuba.sk}}
	}

% TODO:
\date{\small 30. september 2015} % upravte

\begin{document}

\maketitle

\begin{abstract}
Getting the user, the information they seek as quickly and efficiently as possible is key when talking about searching in today’s world. In this article I am choosing to focus on the ways this is achieved behind the scenes. I want to compare different indexing techniques and other optimizations modern database engines employ to provide optimal storage and retrieval efficiency. I also want to look at methods of making retrieval faster by utilizing caching in a way that increases the performance of storage access. Furthermore, querying the database for data the user wants quickly, would not be possible without proper tokenization of the data in the database, so I would like to mention it as well.
\end{abstract}

\section{Introduction}

\lipsum[1]

\section{Indexing}
Cite https://citeseerx.ist.psu.edu/pdf/24ad3ea19602f0737807cb05fafe44c6c2f86aaf part Indexing and the internet - > importance of indexing on the web

Indexing is used to make querying data faster and more efficient. Since the data can be any size, (TODO: how much data search engines store) and stored in any order, going over every record is unfeasible. With the amount of requests users around the globe make, the equipment costs grow rapidly. To solve this issue, indexing has been introduced as a method to help avoid the need of traversing all of the data on each request.

Indices can be created for each of the parameters that needs to be sorted separately. This reduces storage requirements, drastically. (TODO: graphs) Instead of choosing between searching the table each time and having multiple copies of the table ordered by different columns, a third option is now possible. Storing only references to the real table along with the ordered indices. 

Another benefit of indices is their reference to the table is only one directional. In other words the table does not care about how many different indices exist, which provides for easy manipulation and flexibility when iterating over database design.

However, indexing isn't one size fits all, there are many options, which cut different corners and make needed optimizations for even faster querying.

TODO: solarwinds article

TODO: b-trees

\subsection{Primary Index}
\subsection{Secondary Index}
\subsection{Composite Index}
https://www.sciencedirect.com/science/article/abs/pii/0164121288900040
\subsubsection{Covering Index}

\section{Caching}
\lipsum[2]

\section{Tokenization}
\lipsum[1]

\section{Summary}
\lipsum[1]

\newpage

\section*{References}
\href{https://citeseerx.ist.psu.edu/pdf/24ad3ea19602f0737807cb05fafe44c6c2f86aaf}{Database indexing: yesterday and today}\\
\href{https://dl.acm.org/doi/pdf/10.1145/356643.356645}{HASH TABLE M E T H O D S}\\
\href{https://ieeexplore.ieee.org/abstract/document/755618?casa_token=1dPMFNh1uOMAAAAA:EhTd5TzTa1RHsWQThUec5nDGgAe82xXTYaG32s8GsTNty9qoUoSPlu0Rzh-it8kQ-qtl19_LX-0KjA}{Caching on the World Wide Web}\\
\href{https://link.springer.com/chapter/10.1007/978-981-15-6198-6_18}{Study of Various Methods for Tokenization}\\
\href{https://ieeexplore.ieee.org/abstract/document/1540920?casa_token=hzg5FRuCiVQAAAAA:MDNu1j1gc-DR7xTkOo22jTVIlq55BcYoatyx97bKzY3HdvAw_7-4sRhCNF-iXQcbZQTw2Bb3u0YQxg}{Indexing relational database content offline for efficient keyword-based search}
\href{https://books.google.sk/books?hl=en&lr=&id=yQgfCgAAQBAJ&oi=fnd&pg=PP1&dq=relational+database&ots=qPKwl0TFYt&sig=6jOwNojMeS_JzJYt9NTVB7_gJwk&redir_esc=y#v=onepage&q=relational%20database&f=false}{Relational Database Design and Implementation}


%\acknowledgement{Ak niekomu chcete poďakovať\ldots}

% TODO: add articles
% týmto sa generuje zoznam literatúry z obsahu súboru literatura.bib podľa toho, na čo sa v článku odkazujete
%\bibliography{literatura}
%\bibliographystyle{abbrv} % prípadne alpha, abbrv alebo hociktorý iný
\end{document}
