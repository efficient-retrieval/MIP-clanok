% Metódy inžinierskej práce

% TODO twoside
\documentclass[10pt,a4paper]{article}

%\usepackage[slovak]{babel}
%\usepackage[T1]{fontenc}
%\usepackage[IL2]{fontenc} % lepšia sadzba písmena Ľ než v T1
\usepackage[utf8]{inputenc}
\usepackage{graphicx}
\usepackage{url} % príkaz \url na formátovanie URL
\usepackage{hyperref} % odkazy v texte budú aktívne (pri niektorých triedach dokumentov spôsobuje posun textu)
\usepackage{lipsum}

\usepackage{cite}
%\usepackage{times}

\pagestyle{headings}

\title{Getting information to the user optimally\thanks{Semestrálny projekt v predmete Metódy inžinierskej práce, ak. rok 2023/24, vedenie: Meno Priezvisko}} % TODO: meno a priezvisko vyučujúceho na cvičeniach

\author{Samuel Tvrdoň\\[2pt]
	{\small Slovenská technická univerzita v Bratislave}\\
	{\small Fakulta informatiky a informačných technológií}\\
	{\small \texttt{...@stuba.sk}}
	}

% TODO:
\date{\small 30. september 2015} % upravte

\begin{document}

\maketitle

\begin{abstract}
Getting the user, the information they seek as quickly and efficiently as possible is key when talking about searching in today’s world. In this article I am choosing to focus on the ways this is achieved behind the scenes. I want to compare different indexing techniques and other optimizations modern database engines employ to provide optimal storage and retrieval efficiency. I also want to look at methods of making retrieval faster by utilizing caching in a way that increases the performance of storage access. Furthermore, querying the database for data the user wants quickly, would not be possible without proper tokenization of the data in the database, so I would like to mention it as well.
\end{abstract}

\section{Introduction}

\lipsum[1]

\section{Indexing}
\lipsum[1-4]

\section{Caching}
\lipsum[2]

\section{Tokenization}
\lipsum[1]

\section{Summary}
\lipsum[1]

\newpage

\section*{References}
\href{https://citeseerx.ist.psu.edu/pdf/24ad3ea19602f0737807cb05fafe44c6c2f86aaf}{Database indexing: yesterday and today}\\
\href{https://dl.acm.org/doi/pdf/10.1145/356643.356645}{HASH TABLE M E T H O D S}\\
\href{https://ieeexplore.ieee.org/abstract/document/755618?casa_token=1dPMFNh1uOMAAAAA:EhTd5TzTa1RHsWQThUec5nDGgAe82xXTYaG32s8GsTNty9qoUoSPlu0Rzh-it8kQ-qtl19_LX-0KjA}{Caching on the World Wide Web}\\
\href{https://link.springer.com/chapter/10.1007/978-981-15-6198-6_18}{Study of Various Methods for Tokenization}\\
\href{https://ieeexplore.ieee.org/abstract/document/1540920?casa_token=hzg5FRuCiVQAAAAA:MDNu1j1gc-DR7xTkOo22jTVIlq55BcYoatyx97bKzY3HdvAw_7-4sRhCNF-iXQcbZQTw2Bb3u0YQxg}{Indexing relational database content offline for efficient keyword-based search}
\href{https://books.google.sk/books?hl=en&lr=&id=yQgfCgAAQBAJ&oi=fnd&pg=PP1&dq=relational+database&ots=qPKwl0TFYt&sig=6jOwNojMeS_JzJYt9NTVB7_gJwk&redir_esc=y#v=onepage&q=relational%20database&f=false}{Relational Database Design and Implementation}


%\acknowledgement{Ak niekomu chcete poďakovať\ldots}

% TODO: add articles
% týmto sa generuje zoznam literatúry z obsahu súboru literatura.bib podľa toho, na čo sa v článku odkazujete
%\bibliography{literatura}
%\bibliographystyle{abbrv} % prípadne alpha, abbrv alebo hociktorý iný
\end{document}
